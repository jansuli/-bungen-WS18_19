\documentclass[a4paper]{article}
\usepackage{myStyle}
\title{CD HA 01}

\DeclareMathOperator{\gl}{GL}
\begin{document}
\section{}
Die Differentialgleichung zweiter Ordnung
\begin{equation}
    \label{eqn:diff1}
    \dot{\dot x} - μ ( 1 - x^2) \dot x + x = 0, \qquad μ>0,
\end{equation}
ist mit $ξ := \dot x$ äquivalent zum System erster Ordnung
\begin{equation*}
    \left \{ \; \begin{aligned}
    \dot ξ - μ(1-x^2) \dot x + x &= 0,\\
    \dot x &= ξ 
    \end{aligned}\right.
\end{equation*}
bzw.
\begin{equation}
    \label{eqn:diff2}
    \left\{ \; \begin{aligned}
    \dot x &= ξ\\
    \dot ξ &= μ(1-x^2) - x
    \end{aligned} \right.
\end{equation}
Dass kann man der Matlab-Funktion \verb!ode45! übergeben (mit $\mathbf x = (x, ξ) = (x,\dot x)$):
\begin{verbatim}
mu = 1;
t_range = [0,100];
x_init = [0.5, 0.5];
rhs = @(x1, x2) [x2, mu * (1 - x1.^2) - x1];
[t, x] = ode45( @(t,x) rhs(x(1),x(2)), t_range, x_init );

% plotting
X = x(:,1);
Y = x(:,2);
Xsample = X(1:10:end);
Ysample = Y(1:10:end);

arrows = rhs(Xsample, Ysample);

scatter(X,Y, '.b');
hold on
quiver(Xsample, Ysample, arrows(:,1), arrows(:,2), 'r');
hold off
\end{verbatim}

% aufgabe 3
\setcounter{section}{2}
\section{}
\subsection{}\label{aufg:3.1}
Es sei $J$ die Jordan-Normalform von $A$, so dass $A = P^{-1} J P$ für ein $P\in \gl_n(ℝ)$ und
$$J = J_1 + \oplus … \oplus J_k$$
mit den Jordanblöcken
$$J_i = \begin{pmatrix}
λ_i & 1  & 0 & … & … \\
0   & λ_i& 1 & 0 & … \\
    &   \ddots  & \ddots & \ddots & \\
    &    &    & 0 & λ_i
\end{pmatrix} =: D_i + N_i$$
zu einem Eigenwert $λ_i = ξ_i +\overbrace{ \sqrt{-1} }^{=:j} y_i$ von $A$ mit $ξ_i,y_i \in ℝ, ξ_i \ne 0$. $D_i$ ist diagnoal und $N_i$ nilpotent.
Für ein $\symbf x = (x_1, …, x_k)\in ℝ^n$ mit Teilvektoren $x_i$ und einer Gruppierung entsprechend der Größe der Jordanblöcke gilt dann
\begin{align*}
    \| e^{tA} \symbf x \| &= \| P^{-1} e^{tJ} e^{P} \symbf x \| \\
     &= \| P^{-1} \left( e^{tJ_1} \oplus … \oplus e^{tJ_k} \right) \tilde \symbf {x} \| \\
      &= \| P^{-1} \left( e^{tλ_1} e^{N_1} \oplus … e^{tλ_k} e^{N_k0} \right) \tilde{\symbf x} \| \\
    &= \| P^{-1} \left( e^{tξ_i}e^{jty_1} e^{N_1} \oplus … \oplus e^{tξ_k}e^{jtξ_k}e^{N_k}\right) \tilde \symbf x \|\\
    &= \left \|P^{-1}
    \begin{pmatrix}
    e^{tξ_i}e^{jty_1} e^{N_1} \tilde x_1\\
    \vdots \\
    e^{tξ_k}e^{jtξ_k}e^{N_k} \tilde x_k
    \end{pmatrix}\right\|
\end{align*}
Da $\symbf x\ne 0$ ist $\tilde \symbf x\ne 0$ und es gibt im rechten Vektor mindestens eine Komponente $\ne 0$. Weil $P^{-1}$ unabhängig von $t$ ist, gilt also
$$\| e^{tA} \symbf x\| \to \infty \text{ für $t\to \pm \infty$}.$$
(Betrachte z.B. die Maximumsnorm.)

\subsection{}
Es sei $e^{tA}$ hyperbolisch.
Angenommen, es gibt eine nichttriviale periodische Lösung 
$$x(t) = e^{tA}x_0, \quad x_0\in ℝ^n\smallsetminus \{0\}.$$
Dann ist der periodische Orbit 
$$γ = \{ x(t) : t\in ℝ \} = x( [0,T] )$$
als Bild eines Kompaktums unter der stetigen Abbildung $x$ insbesondere beschränkt, es gibt also ein $M\ge 0$, so dass
$$\| x(t) \| = \| e^{tA} x_0 \| \le M \qquad  ∀t\in ℝ.$$
Nach \ref{aufg:3.1} geht aber $ \| e^{tA} x_0 \| \to \infty$ für $t \to \infty$ oder $t\to -\infty$. Dies gelte o.B.d.A. für $t\to \infty$. Dann gibt es also ein $τ\in ℝ$, so dass
$$\| e^{tA} x_0 \| \ge M \qquad ∀t\ge τ$$
im Widerspruch zur Beschränktheit des Orbits.
\end{document}