\documentclass[a4paper]{article}
\usepackage{myStyle}
\title{CD HA 01}

\DeclareMathOperator{\gl}{GL}
\begin{document}
\section{}
Die Differentialgleichung zweiter Ordnung
\begin{equation}
    \label{eqn:diff1}
    \ddot{x} - μ ( 1 - x^2) \dot x + x = 0, \qquad μ>0,
\end{equation}
ist mit $ξ := \dot x$ äquivalent zum System erster Ordnung
\begin{equation*}
    \left \{ \; \begin{aligned}
    \dot ξ - μ(1-x^2) ξ + x &= 0,\\
    \dot x &= ξ 
    \end{aligned}\right.
\end{equation*}
bzw.
\begin{equation}
    \label{eqn:diff2}
    \left\{ \; \begin{aligned}
    \dot x &= ξ\\
    \dot ξ &= μ(1-x^2)ξ - x
    \end{aligned} \right.
\end{equation}
Dass kann man der Matlab-Funktion \verb!ode45! übergeben (mit $\mathbf x = (x, ξ) = (x,\dot x)$):
\begin{verbatim}
mu = 1;
t_range = [0,100];
x_init = [0.5, 0.5];
rhs = @(x1, x2) [x2; mu * (1 - x1.^2) - x1];
[t, x] = ode45( @(t,x) rhs(x(1),x(2)), t_range, x_init );

% plotting
X = x(:,1);
Y = x(:,2);

plot3(X,Y,t,'b');
hold off
\end{verbatim}


\section{}
\subsection{}
Es gilt das Matrixexponential $e^{tB}$ für $B=\begin{pmatrix}
0&1\\-1&0
\end{pmatrix}$ zu berechnen.
Man stellt fest, dass
\begin{align*}
B^2 &= \begin{pmatrix}
0&1\\-1&0
\end{pmatrix} \begin{pmatrix}
0&1\\-1&0
\end{pmatrix} = \begin{pmatrix}
-1&0\\0&-1
\end{pmatrix} =: -I,\\
B^3 &= B^2 \cdot B = \begin{pmatrix}
-1&0\\0&-1
\end{pmatrix} \begin{pmatrix}
0&1\\-1&0
\end{pmatrix} = \begin{pmatrix}0&-1\\1&0\end{pmatrix} = -B,\\
B^4 &= (B^2)^2 =  \begin{pmatrix}
-1&0\\0&-1
\end{pmatrix} \begin{pmatrix}
-1&0\\0&-1
\end{pmatrix} = \begin{pmatrix}
1&0\\0&1\end{pmatrix} = I,\\
B^5 &= B
\end{align*}
Also
\begin{align*}
e^{tB} &= \sum_{k=0}^∞ \frac{B^k}{k!}\\
&= \sum_{k \in 0+4ℕ} \frac{t^kI}{k!} + \sum_{k\in 1+4ℕ} \frac{t^kB}{k!} + \sum_{k\in 2+ 4ℕ} \frac{t^k (-I)}{k!} + \sum_{k\in 3+4ℕ} \frac{t^k-B}{k!}\\&=
\sum_{k=0}^\infty \frac{t^{4k}I}{(4k)!} + \sum_{k=0}^∞ \frac{t^{4k+1}B}{(4k+1)!} +\sum_{k=0}^∞ \frac{t^{4k+2}(-I)}{(4k+2)!} +\sum_{k=0}^∞\frac{t^{4k+3}(-B)}{(4k+3)!}\\
&= \sum_{k=0}^∞ \left(\frac{t^{4k}}{(4k)!} - \frac{t^{4k+2}}{(4k+2)!} \right)I +
\sum_{k=0}^∞ \left(\frac{t^{4k+1}}{(4k+1)!} - \frac{t^{4k+3}}{(4k+3)!} \right)B\\
&= \sum_{k=0}^∞ \frac{(-1)^k t^{2k}}{(2k)!} I +\sum_{k=0}^∞ \frac{(-1)^kt^{2k+1}}{(2k+1)!} B\\
&=\cos(t)I + \sin(t) B\\
&=\begin{pmatrix}
\cos t & \sin t\\
-\sin t & \cos t
\end{pmatrix} 
\end{align*}
\subsection{}
Es gilt
\begin{align*}
e^B e^C &= \left( \sum_{k=0}^∞ \frac{B^k}{k!} \right)\left( \sum_{k=0}^∞ \frac{C^k}{k!} \right)\\
&= \sum_{n=0}^∞ \sum_{k=0}^n \frac{B^k}{k!} \frac{C^{n-k}}{(n-k)!}\\
&= \sum_{n=0}^∞\frac{1}{n!} \sum_{k=0}^n \frac{n!}{k!(n-k)!} B^kC^{n-k}\\
&= \sum_{n=0}^∞\frac{1}{n!} \sum_{k=0}^n \binom{n}{k} B^k C^{n-k} 
\end{align*}
\emph{Kommutieren} die Matrizen, ist der binomische Lehrsatz anwendbar und es gilt
$$e^B e^C = \sum_{n=0}^∞ \frac{1}{n!}\sum_{k=0}^n \binom{n}{k} B^k C^{n-k} = \sum_{n=0}^∞ (B+C)^n = e^{B+C}.$$
% aufgabe 3
\setcounter{section}{2}
\section{}
\subsection{}\label{aufg:3.1}
Es sei $J$ die Jordan-Normalform von $A$, so dass $A = P^{-1} J P$ für ein $P\in \gl_n(ℂ)$ und
$$J = J_1 + \oplus … \oplus J_k$$
mit den Jordanblöcken
$$J_i = \begin{pmatrix}
λ_i & 1  & 0 & … & … \\
0   & λ_i& 1 & 0 & … \\
    &   \ddots  & \ddots & \ddots & \\
    &    &    & 0 & λ_i
\end{pmatrix} =: D_i + N_i$$
zu einem Eigenwert $λ_i = ξ_i +\overbrace{ \sqrt{-1} }^{=:j} y_i$ von $A$ mit $ξ_i,y_i \in ℝ, ξ_i \ne 0$. $D_i$ ist diagnoal und $N_i$ nilpotent.
Für ein $\symbf x = (x_1, …, x_k)\in ℝ^n$ mit Teilvektoren $x_i$ und einer Gruppierung entsprechend der Größe der Jordanblöcke gilt dann
\begin{align*}
    \| e^{tA} \symbf x \| &= \| P^{-1} e^{tJ} e^{P} \symbf x \| \\
     &= \| P^{-1} \left( e^{tJ_1} \oplus … \oplus e^{tJ_k} \right) \tilde{\symbf {x}} \| \\
      &= \| P^{-1} \left( e^{tλ_1} e^{N_1} \oplus … e^{tλ_k} e^{N_k} \right) \tilde{\symbf x} \| \\
    &= \| P^{-1} \left( e^{tξ_i}e^{jty_1} e^{N_1} \oplus … \oplus e^{tξ_k}e^{jtξ_k}e^{N_k}\right) \tilde{\symbf x} \|\\
   &= \left \|P^{-1}
    \begin{pmatrix}
    e^{tξ_1}e^{jty_1} e^{N_1} \tilde x_1\\
    \vdots \\
    e^{tξ_k}e^{jtξ_k}e^{N_k} \tilde x_k
    \end{pmatrix}\right\|
\end{align*}
Da $\symbf x\ne 0$ und $P\in \gl_n(ℂ)$ ist $\tilde{\symbf x}\ne 0$ und es gibt im rechten Vektor mindestens eine Komponente $\ne 0$. Weil $P^{-1}$ unabhängig von $t$ ist, gilt also
$$\| e^{tA} \symbf x\|_∞ \to \infty \text{ für $t\to \pm \infty$},$$
da hier $e^{tξ_i}$ explodiert und während $|e^{jty_i}|=1$.

\subsection{}
Es sei $e^{tA}$ hyperbolisch.
Angenommen, es gibt eine nichttriviale periodische Lösung 
$$x(t) = e^{tA}x_0, \quad x_0\in ℝ^n\smallsetminus \{0\}.$$
Dann ist der periodische Orbit 
$$γ = \{ x(t) : t\in ℝ \} = x( [0,T] )$$
als Bild eines Kompaktums unter der stetigen Abbildung $x$ insbesondere beschränkt, es gibt also ein $M\ge 0$, so dass
$$\| x(t) \| = \| e^{tA} x_0 \| \le M \qquad  ∀t\in ℝ.$$
Nach \ref{aufg:3.1} geht aber $ \| e^{tA} x_0 \| \to \infty$ für $t \to \infty$ oder $t\to -\infty$. Dies gelte o.B.d.A. für $t\to \infty$. Dann gibt es also ein $τ\in ℝ$, so dass
$$\| e^{tA} x_0 \| \ge M \qquad ∀t\ge τ$$
im Widerspruch zur Beschränktheit des Orbits.
\end{document}