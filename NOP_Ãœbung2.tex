\documentclass[a4paper]{article}
\usepackage{myStyle}
\title{CD HA 01}

\DeclareMathOperator{\wto}{\rightharpoonup}
\begin{document}
\section{Schwache Folgenstetigkeit}
Es seien $U,V$ reelle Banachräume und $F\colon U \to V$ sei stetig und linear.
$U^*$ bezeichne den Dualraum von $U$, also die Menge der linearen, stetigen Abbildungen $U \to ℝ$. 
Analog sei $H^*$ der Dualraum von $V$.
Ferner sei $(u_n)$ schwach konvergent in $U$. Genau dann gilt also
\begin{equation}
\label{eqn:def1}
\lim_{n\to ∞} φ(u_n) = φ(u) \qquad ∀φ\in U^*.
\end{equation}
Da $F$ stetig und linear ist, ist für eine beliebiges $ψ\in V^*$ auch $ψ\circ F : U \to ℝ$ stetig und linear, also $ψ\circ F \in U^*$. Nach \eqref{eqn:def1} gilt damit
\begin{equation}
\label{eqn:folge1}
\lim_{n\to ∞} ψ(F(u_n)) = ψ.
\end{equation}
Da in \eqref{eqn:folge1} $ψ$ beliebig war folgt tatsächlich $F(u_n) \wto F(u)$.

\section{Strenge Konvexität}
Es sei $g(u) := \| u\|^2_U$ eine Abbildung vom Hilbertraum $(U,\|•\|_U)$ in die reellen Zahlen. Es seien $x,y \in U, x\ne y$ und $0<λ<1$. Dann gilt
\begin{align*}
g(λ x + (1-λ)y) &= \| λx + (1-λ) y\|_U^2 \\
&= \langle λx + (1-λ) y, λx + (1-λ) y \rangle \\
&\stackrel{\text{Bilinearität}}= λ\langle x, λx + (1-λ) y \rangle + (1-λ) \langle y , λx + (1-λ) y \rangle \\
&= λ^2 \langle x, x \rangle + 2λ(1-λ) \langle x, y \rangle + (1-λ)^2 \langle y, y \rangle  .
\end{align*}
Die zu beweisende Ungleichung ist also äquivalent zu
\begin{align*}
&\phantom{⇔~} λ^2 \langle x, x \rangle + 2λ(1-λ) \langle x, y \rangle + (1-λ)^2 \langle y, y \rangle < λ \langle x, x\rangle + (1-λ)\langle y,y\rangle \\
&⇔ λ \langle x, x \rangle + 2(1-λ) \langle x, y \rangle + \frac{(1-λ)^2}{λ} \langle y, y\rangle < \langle x, x \rangle + \frac{1-λ}{λ} \langle y,y \rangle \\
&⇔\frac{λ}{1-λ} \langle x, x \rangle + 2 \langle x, y \rangle + \frac{1-λ}{λ}\langle y,y \rangle < \frac{1}{1-λ}\langle x,x \rangle + \frac{1}{λ} \langle y, y\rangle\\
&⇔2\langle x,y\rangle < \langle x, x \rangle + \langle y,y\rangle \\
&⇔0 < \langle x, x-y \rangle + \langle y-x, y \rangle = \langle x-y , x-y \rangle
\end{align*}
und das gilt offensichtlich wegen der Definitheit des Skalarprodukts und $x\ne y.$

\subsection*{Zusatz}
Sei $(Y, \|•\|_Y)$ ein weiterer Hilbertraum und $S\colon U\to V$ ein linearer stetiger Operator sowie $y_d \in Y, κ>0$ gegeben.
Dann ist $f(u): = \| Su-y_d \|^2_Y + κ \| u \|^2_U$ konvex, denn für $0<λ<1$ gilt
\begin{align*}
\|S(λx + (1-λ)y) - y_d \|^2_Y + κ\| λx + (1-λ)y \|^2_U &=
\| λ Sx + (1-λ) Sy - y_d \|^2_Y + κ\| λx + (1-λ)y \|^2_U\\
&\hspace{-2em}= \| λ Sx -λ y_d + (1-λ) Sy -(1-λ) y_d \|^2_Y + κ\| λx + (1-λ)y \|^2_U\\
&= \| λ (Sx - y_d) + (1-λ) (Sy-y_d) \|_Y^2 + κ\| λx + (1-λ)y \|^2_U\\
&\stackrel{(a)}< λ \left( \| Sx-y_d \|_Y^2 + κ \|x\|_U^2 \right) + (1-λ) \left( \| S_y - y_d\|_Y^2 + κ \| y \|_U^2 \right) \\
&= λf(x) + (1-λ) f(y)
\end{align*}

\section{Endlichdimensionaler Banachraum}
\colorbox{gray!20}{\parbox{\textwidth}{Ein Banachraum $X$ ist genau dann endlichdimensional, wenn die abgeschlossene Einheitskugel in $X$ kompakt ist.}}

\begin{proof}
Es sei $\mathcal B = \{x\in X: \|x \|_X \le 1 \}$ die Einheitskugel. Ist $X$ endlichdimensional, so ist $\mathcal B$ kompakt nach \emph{Heine-Borel}, da $\mathcal B$ abgeschlossen und beschränkt ist.

\textbf{Bessere Lösung}: \\
\href{https://de.wikiversity.org/wiki/Kompaktheitssatz_von_Riesz}{Wikiversity - Kompaktheitssatz von Riesz}\\
Alternativ kann meine Lösung auch abgewandelt werden zu:
\href{https://www.uni-due.de/~adf040p/skripte/FunktAnSkript15.pdf}{Lemma 3.9}\\
Sei andererseits $\mathcal B$ kompakt. Für einen Widerspruch nehmen wir an, $X$ wäre undendlich-dimensional.
Per Definition hat jede offene Überdeckung von $\mathcal B$ eine endliche Teilüberdeckung, die $\mathcal B$ enthält. 
Für eine solche endliche Teilüberdeckung gibt es dann einen endlich-dimensionalen, heißt \emph{echten}, Unterraum $U\le V$, der $\mathcal B$ enthält. Da $U$ endlich-dimensional ist, ist $U$ abgeschlossen und das Lemma kann angewendet werden.
Man findet also für jedes $0<δ<1$ ein $x\in \mathcal B$, so dass $d(x,U) \ge 1-δ>0$ im Widerspruch zu $\mathcal B\subseteq U$.
\end{proof}

\section{Konvexität und schwache Folgenabgeschlossenheit}
Es sei $M\subset X$ konvex und abgeschlossen. 
Ferner sei $(x_n)\subseteq M\subset K$ eine Folge und $x_n\wto x^*$, also
\begin{equation}\label{eqn:def4}\lim_{n\to ∞} \langle x_n, w \rangle = \langle x^* , w \rangle\qquad ∀w\in X'.\end{equation}
Angenommen, $x^*\notin M$. 
Dann gibt es nach Trennungssatz ein $w'\in X'$ und ein $α\in ℝ$, so dass
$$\Re \langle x_n, w' \rangle \le α \; ∀ n \text{ und }\Re \langle x^*, w' \rangle > α.$$  
Dann erhält man mit \eqref{eqn:def4} einen Widerspruch:
$$α \ge \lim_{n\to ∞} \Re \langle x_n, w'\rangle = \Re \langle x^*, w' \rangle > α.$$

Also muss $x^*\in M$ sein.
\end{document}