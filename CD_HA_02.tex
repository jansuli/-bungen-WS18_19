\documentclass[a4paper]{article}

\usepackage{myStyle}

\title{CD HA 02}

\begin{document}

\section{Invariante Unterräume}
Die Matrix
$$A_1 = \begin{pmatrix}
\frac{2}{5} & -\frac{6}{5} \\
-\frac{6}{5} & -\frac{7}{5}
\end{pmatrix}$$
hat die Eigenwerte $λ_1 = -2$ und $λ_2 = 1$ und es gilt
\begin{align*}
E^s = \ker (A - λ_1 I) &= \ker \begin{pmatrix}
12/5 & -6/5 \\ -6/5 & 3/5
\end{pmatrix} = \ker \begin{pmatrix}
1 & -6/12 \\ 1 & -3/6
\end{pmatrix}\\
&=\ker\begin{pmatrix}
1 & -6/12\\
0 & 0
\end{pmatrix} = \left\langle \begin{pmatrix}
1 \\ 12/6
\end{pmatrix} \right\rangle  
= \left\langle \begin{pmatrix}
1 \\ 2
\end{pmatrix} \right\rangle = \langle v_1 \rangle.
\end{align*}
Analog findet man
\begin{align*}
E^u = \ker (A -λ_2 I) = \langle v_2 \rangle = \left\langle \begin{pmatrix}
-2\\ 1
\end{pmatrix}\right\rangle.
\end{align*}


% Aufgabe 2
\section{Eigenwerte Poincaré-Differential}
Es gilt zu zeigen, dass die Eigenwerte der Jacobischen der Poincaré-Abbildung unabhängig von der Wahl des transversalen Schnittes sind. 
Dabei nutzt man, dass ähnliche Matrizen die gleichen Eigenwerte haben.

Zunächst stellt man aber fest, dass verschiedene Poincaré-Abbildungen zueinander konjugiert sind. 
Es seien nämlich $Σ_1, Σ_2$ verschiedene transversale Schnitte durch zwei (nicht notwendig verschiedene) Punkte $p_1$ und $p_2$ auf dem periodischen Orbit $γ$. 
Seien $P_1, P_2$ die zugehörigen Poincaré-Abbildungen.
Dann gilt
$$p_2 = φ^s (p_1) \text{ für ein $0\le s < T$.}$$
Das Schnittlemma/Satz über implizite Funktionen liefert einen lokalen Diffeomorphismus
$$h\colon Σ_1 \to Σ_2, \; h(p_1) = p_2,$$
gegeben durch $h(ξ) = φ^{τ(ξ)} (ξ)$.
Damit gilt dann auch
$$P_2 \circ h = h\circ P_1.$$

Differentiation dieser Gleichung liefert
$$DP_2(p_2) \cdot Dh(p_1) = Dh(p_1) \cdot DP_1(p_1).$$
Da $h$ diffeomorph ist, können wir invertieren und mit $T = Dh(p_1)$ gilt:
$$DP_2(p_2) = T DP_1 (p_1) T^{-1},$$
also sind die Jacobi-Matrizen ähnlich und die Eigenwerte gleich, also unabhängig von der Wahl der Punkte auf dem Orbit und der transversalen Schnitte.
\end{document}