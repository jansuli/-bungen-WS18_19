\documentclass[a4paper]{article}

\usepackage{myStyle}
\title{NILINOP HA 03}

\newcommand{\wto}{\rightharpoonup}
\newcommand{\xwto}[1]{\stackrel{#1}\rightharpoonup}

\begin{document}
\section{Schwach unterhalb-stetig}
Es seien $X,Y$ Banachräume und es sei $F:X\to ℝ$ schwach unterhalbstetig, d.h. für alle Folgen $(x_k)\subseteq X$ mit $x_k\xwto{k\to ∞}x \in X$ gilt
$$\liminf_{k\to ∞} F(x_k) \ge F(x).$$

Sei jetzt $(x_k)$ beliebig mit $x_k\wto x$.
\begin{enumerate}
\item $α\ge 0$, dann gilt wegen der Nicht-Negativität und Eigenschaften des $\liminf$ direkt
$$\liminf_{k\to ∞} aF(x_k) \ge aF(x).$$
\item Sei $G\colon X\to ℝ$ schwach unterhalb stetig. Dann gilt
\begin{align*}
\liminf_{k\to ∞} (F+G)(x_k) &= \liminf_{k\to ∞} F(x_k) + G(x_k) \\
&\stackrel{\text{Superadd.}}\ge \liminf_{k\to ∞} F(x_k) + \liminf_{k\to ∞} G(x_k) \\
&\ge F(x) + G(x) = (F+G)(x).
\end{align*}
\item Sei $φ\colon ℝ\to ℝ$ schwach unterhalbstetig und monoton wachsend. Dann gilt
\begin{align*}
\liminf_{k\to ∞} φ\circ F (x_k) &= \liminf_{k\to ∞} φ\left( F(x_k)\right) \\
&\stackrel{\text{$φ$ m.w.}} = \lim_{k\to ∞} φ \left( \inf_{m\ge k} F(x_m) \right) \\
&\ge \liminf_{k\to ∞} φ \left( \inf_{m\ge k} F(x_m) \right)\\
&\stackrel{\text{$φ$ s.u.s}}\ge φ\left(\liminf_{k\to ∞} F(x_k) \right) \\
&\stackrel{\text{$φ$ m.w.}}\ge φ(F(x)).
\end{align*}
\item Sei $φ\colon Y\to X$ schwach-folgenstetig und $y_k\wto y\in Y$. Dann gilt $φ(y_k) \wto φ(y) \in Y$ und
\begin{align*}
\liminf_{k\to ∞} F(φ(y_k)) \ge F \left( φ(y) \right).
\end{align*}
\item Sei $I$ eine Indexmenge und $F_i\colon X\to ℝ$ schwach unterhalbstetig. Dann gilt für 
$$ψ\colon X\to ℝ, \; ψ(x) = \sup_{i\in I}F_i(x),$$
dass
\begin{align*}
\liminf_{k\to ∞} ψ(x_k) &= \liminf_{k\to ∞} \sup_{i\in I} F_i (x_k)\\
&= \lim_{k\to ∞} \inf_{m\ge k} \sup_{i\in I} F_i(x_m)\\
&\ge \lim_{k\to ∞} \sup_{i\in I} \inf_{m\ge k} F_i(x_m)\\
&= \sup_{i\in I} \liminf_{k\to ∞} F_i(x_k)\\
&\ge \sup_{i\in I} F_i (x) = ψ(x).
\end{align*}
\end{enumerate}

\section{}
Noch zu tun

\section{}
Sei $X$ normierter Raum, $M\subset X$ konvex und nicht-leer sowie $f\colon M\to ℝ$. 
Ist für jedes  $α\in ℝ$ die Menge $M_α = \{ x\in M:f(x) \le α\}$ konvex, so heißt $f$ quasikonvex.
Es gilt nun zu zeigen, dass
$$f\colon ℝ\to ℝ, f(x) =xe^x$$
quasikonvex ist.

\begin{proof}
Sei $α\in ℝ$ und
$$M_α = \{x\in ℝ: xe^x \le α\}.$$
Für $α \ge -\frac{1}{e}$ ist die Lambert'sche W-Funktion definiert und $M_α \ne ∅$.
In diesem Fall gilt für $x,y\in M_α$ mit
$$xe^x = ξ \le α, \; ye^y = υ\le α,$$
dass
$$W(λξ + (1-λ) υ)$$
\end{proof}
\end{document} 
