\documentclass{article}
\usepackage{myStyle}

\title{Mod Übung 2}
\begin{document}

\setcounter{section}{1}
\section{}
\subsection{}
Gegeben ist die intensionale Darstellung der Relation
$$R := \{ (x,y) \in ℕ^2: y=x^2 \wedge x \le 6 \}.$$
Dann ist die extensionale Darstellung
$$R = \{(1, 1), (2,4), (3,9), (4, 16), (5,25), (6,36) \}.$$

\begin{itemize}
\item $R$ ist nicht reflexiv wegen $(2,2)\notin R$.
\item $R$ ist nicht irreflexiv wegen $1R1$.
\item $R$ ist nicht symmetrisch.
\item $R$ ist antisymmetrisch da $(1,1)$ das einzige Tupel mit $aRb$ und $bRa$ ist.
\item $R$ ist nicht transitiv, denn für $(x,y),(y,z)\in R$ gilt i.A. nicht
$z = x^2.$
\item $R$ ist nicht alternativ, es gibt $(a,b)\in ℕ^2$ mit $(a,b),(b,a)\notin R$.
\item $R$ ist nicht asymmetrisch, da $(1,1)\in R$.
\end{itemize}

\subsection{}
$$\{\text{Anja, Horst, Dieter} \} \times M$$
mit 
$$M = \left \{ (A,B):A, B \in \mathrm{Pow}(\text{Wochentage}), A\cap B = ∅, A\cup B=\text{Wochentage} \right\}.$$
Dann modelliert ein Tupel der Form (Name, Uni-Tage, Abwesenheitstage) den Sachverhalt. 
\subsection{}
Eine Relation ist eine Teilmenge von $M\times N$ und es gibt $2^{mn}$ solche Teilmengen, da
$$| \mathrm{pow}(M\times N)| = 2^{mn}.$$

% Aufgabe
\section{}
\subsection{} Für die Relation $R = \{ (a,b) \in ℕ^2 : a\le b\}$ gilt zum Beispiel
$(1,2),(2,2)\in R$, aber $1\ne 2$, also ist $R$ keine Funktion.

\subsection{}
\begin{center}
\begin{tabular}{c|c|c|c|c|c}
 & total & partiell & injektiv & surjektiv & bijektiv \\ \hline 
$f\colon ℕ\to ℝ,f(x) = x^2$ & ja & nein & ja &nein \\ 
$f\colon ℝ\to ℝ^{\ge 0}, f(x) = x^2$ & ja & nein & nein, da $x^2 = (-x)^2$ &ja &\\
$f\colon ℝ\to ℝ^{\ge 0}, f(x) = \sqrt{x}$ &  nein, da $B \ne ℂ$ & ja & ja & ja &\\
$f\colon ℝ\to ℝ, f(x) = \frac{1}{x}$ & ja & nein& ja & nein, 0 erwischt man nicht &
\end{tabular}
\end{center}

\setcounter{section}{4}
%Aufgabe5
\section{}
\subsection{}
Falsch, mit $α = A$ ist die einzige erfüllende Interpretation $A\mapsto \text{wahr}$ und das einzige Atom $A$ ist also mit $w$ bewertet.
\subsection{}
Wahr, eine Tautologie ist unter jeder Bewertung erfüllt.
\subsection{}
Wahr, denn es gelte $α\approx β$ (logisch äquivalent), und $I$ eine Bewertung. 
Es gilt
$$(\neg α) = w  ⇔ I(α) = f,\; I(\neg β) = w ⇔ I(β) = f.$$
Also $I(α) = I(β) ⇔ I(\neg α) = I(\neg β)$
Wahr, 
\end{document}